\chapter{Generalit\`a}
{\bf \emph{OpenChef}} \`e un software di supporto di gestione, monitoraggio e 
ottimizzazione delle attivita di ristorazione, approvvigionamento e contabilit\`a.
  
Deve:
\begin{itemize}
      \item Consentire all'operatore di effettuare un'ordinazione su specifiche del cliente.
      \item Gestire il flusso delle ordinazioni da evadere in cucina in base alle priorit\`a 
          e alle specifiche richieste.
      \item Consentire l'intervento dell'operatore per la rettifica delle ordinazioni.
      \item Calcolare il conto di un tavolo.
      \item Effettuare prenotazioni.
      \item Redazione di bilancio.
 \end{itemize}

\section{Prospettive del Prodotto}
    
{\bf \emph{OpenChef}} \`e costituito da un insieme di palmari wireless, interconnessi 
con il  sistema centrale e utilizzati dagli operatori.
Il sistema centrale, regolamenta tutte le azioni per le prenotazioni, gli ordini, 
le code pietanze e la cucina, il magazzino, la cassa e la contabilit\`a.

\subsection{Interfaccia Utente}

I palmari wireless consentono all'operatore tramite le informazioni acquisite dall'OpenChef 
di individuare e  segnalare elementi necessari per gestire in modo efficace ed efficiente 
il monitoraggio delle code pietanze e garantire la corretta esecuzione degli ordini.

\subsubsection{Interrogazione del sistema dal Customer}

Visualizzazione delle modifiche d'ordine, stato dell'ordine nella coda pietanze e residuo 
tempo di preparazione di un'ordine.

\subsection{Interfaccia Hardware}

L'elaboratore centrale \`e dotato di monitor, tastiera, mouse, unit\ centrale 
(con disco rigido, memoria, processore, lettore/masterizzatore cd/dvd, scheda di rete), 
collegamento al sistema centrale tramite rete locale.

A discrezione del committente il numero necessario di palmari da interagire con 
l'elaboratore centrale.

\subsection{Intefaccia Software}

Operativit\`a del software su tutte le piattaforme hardware supportate dai sistemi operativi: 
\begin{itemize}
    \item Microsoft Windows (Xp e superiori) 
    \item Linux (distribuzioni Debian, Ubuntu e Redhat) 
    \item Macintosh (Panther e superiori)
\end{itemize}

\subsection{Intefaccia delle Comunicazioni}

L'elaboratore centrale e i palmari sono connessi via wireless rispondente alle specifiche 802.11 n.

\subsection{Vincoli di Memoria}

{\bf Elaboratore centrale}:
\begin{itemize}
    \item 4 Gb di RAM
    \item 320 Gb HDD (possibilit\`a di cloud computing)
\end{itemize}
{\bf Palmare}:
\begin{itemize}
    \item Da 64 a 128 Mb di RAM
    \item Memoria interna da 2 Gb
\end{itemize}

\section{Funzionalit\`a del Prodotto}
{\bf \emph{OpenChef}} \`e costituito da una serie di palmari collegati in rete wireless 
con l'elaboratore centrale che automaticamente gestir\`a tutte le operazioni di routine 
di coda delle pietanze, controllo delle scorte di magazzino, ordini evasi, cassa 
e contabilit\`a.

\subsubsection{Il sistema software deve gestire}

\begin{itemize}
    \item L'effetuazione di nuovi ordini mediante una interfaccia e consentire l'interrogazione 
        per la disponibilit\`a dei prodotti necessari. Deve essere possibile definire modifiche 
        dal menu standard e priorit\`a particolari.
    \item Il controllo delle pietanze in coda. Sulla base delle informazioni ottenute dal 
        sistema centrale, l'operatore deve poter configurare opportunamente i vari ordini decidendo quali 
        favorire.
    \item Stima in tempo reale del coda delle pietanze.
    \item Proiezione della situazione degli ordini in un intervallo temporale su tutta l'attivit\`a 
        del ristorante.
    \item Registrazione di informazioni e dati che indicano i giorni con i livelli massimi di 
        utenza e i tempi di preparazione per le varie pietanze  che possono essere visualizzati 
        su palmari per la consultazione dell'operatore.
\end{itemize}

\section{Caratteristiche Utente}
Gli utenti destinati all'uso del prodotto sono:
\begin{itemize}
    \item {\bf Operatore} deve poter effettuare prenotazioni, effettuare ordini, modifiche d'ordine, 
        richiesta conto, visione della code pietanze, cancellazione ordine, visione del magazzino.
    \item {\bf Cliente} si relaziona al sistema tramite l'operatore.
\end{itemize}

\section{Vincoli}
    \subsection{Limitazioni hardware}
        Non \`e possibile calcolare effettuare un numero di prenotazioni superiori al numero 
        dei palmari operatore disponibili.
    \subsection{Interfaccia con le altre applicazioni}
        ...
    \subsection{Requisiti di linguaggio ad alto livello}
        ...
    \subsection{Considerazoni sulla sicurezza}
        Il sistema consente l'accesso  all'operatore tramite operazione di login che gli permette 
        l'accesso al sistema per effettuare le operazioni previste.
