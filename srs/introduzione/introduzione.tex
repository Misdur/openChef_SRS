\chapter{Introduzione}
\section{Obiettivo}
    L'obiettivo di questo SRS, \`e illustrare i requisiti di un sistema 
    software per l'amministrazione e il monitoraggio di tutte le attivit\`a 
    di un ristorante con l'ausilio di mezzi elettronico-informatico.
    Gli utenti del sistema sono:
    \begin{itemize}
        \item {\bf Operatore} colui che interagisce con il sistema per gestire, 
            impostare e reimpostare  le informazioni e le operazioni 
            all'interno del ristorante.
    \end{itemize}
    I  destinatari del sistema sono:
    \begin{itemize}
        \item {\bf Operatore} colui che preleva le informazioni dal sistema per 
            monitorare le attivit\`a del ristorante.
    \end{itemize}
    
\section{Scopo}
    Il sistema software OpenChef permette il monitoraggio dell'intera attivit\`a 
    del ristorante, il controllo delle scorte di magazzino, fornisce le informazioni 
    relative alle attivit\`a in corso nel ristorante (in tempo reale e in termini 
    statistici), supporta l'amministrazione e la revisione contabile, ed elabora 
    in maniera ottimale gli ordini di utenza con l'ausilio di strumenti wireless. 

\section{Definizioni, acronimi e abbreviazioni}
    \begin{itemize}
        \item {\bf Palmare}: strumento wireless che permette l'interazione dei 
            singoli operatori (camerieri, cuochi, cassiere, ecc...) col sistema.
        \item {\bf Coda Pietanze}: Lista delle ordinazioni ancora in attesa 
            per la preparazione.
        \item {\bf Ordine Evaso}: ordine la cui elaborazione in cucina ha gi\`a 
            avuto inizio.
        \item {\bf Ordine Pronto}: ordine in attesa di consegna al tavolo.
    \end{itemize}

\section{Riferimenti}
    \begin{description}
        \item [-] ISO 12207
        \item [-] ISO 9126
        \item [-] ISO 22005
    \end{description}

\section{Struttura Generale}
    Il documento SRS \`e suddiviso nelle seguenti sezioni:
    \begin{enumerate}
        \item Introduzione e definizione di presupposti, scopo, acronimi e 
            abbreviazioni, referenze.
        \item Descrizione generale, prospettive e funzionamento del prodotto, 
            caratteristiche di utilizzo e vincoli generali.
        \item Specifiche dei requisiti.
        \item Appendice.
    \end{enumerate}    
