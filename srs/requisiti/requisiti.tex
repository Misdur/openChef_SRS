\chapter{Specifiche dei Requisiti}
\section{Casi d'Uso}
\begin{enumerate}
    \item {\bf Caso d'uso}: \uline{{\bf Ordinazione Tavolo}}
    \begin{itemize}
        \item {\bf Attore}: Operatore
        \item {\bf Input}: Pietanze, Coperti, Numero Tavolo
        \item {\bf Precondizioni}: Postazione operatore connessa alla rete
        \item {\bf Output}: Presa carico dell'ordine dalla cucina
        \item {\bf Postcondizioni}: Conferma dell'ordine all'utente
        \item {\bf Descrizione scenari}:
        \begin{itemize}
            \item \uline {Scenario 1}
            \begin{itemize}
                \item Il cliente effettua un ordine all'operatore
                \item L'operatore inserisce nel sistema pietanze, bevande e 
                    numero tavolo
                \item Se tutto disponibile, l'operatore lo comunica al cliente
                \item Il cliente conferma l'ordine
                \item Il cliente attende l'evasione dell'ordine
            \end{itemize}
            \item \uline {Scenario 2}:
            \begin{itemize}
                \item Il cliente effettua un ordine all'operatore
                \item L'operatore inserisce nel sistema pietanze, bevande e numero tavolo
                \item Una o pi\`u pietanze non \`e disponibile
                \item L'operatore segnala la pietanza mancante al sistema centrale
                \item L'operatore comunica l'impossibilit\`a di evadere l'ordine richiesto al cliente
            \end{itemize}
            \item \uline{Scenario 3}:
            \begin{itemize}
                \item Il cliente effettua un ordine all'operatore
                \item L'operatore inserisce nel sistema pietanze, bevande e numero tavolo
                \item Una o pi\`u pietanze non \`e presente nel database
                \item L'operatore segnala l'errore del database al sistema centrale
                \item L'operatore comunica l'impossibilit\`a di evadere l'ordine richiesto al cliente
            \end{itemize}
        \end{itemize}
    \end{itemize}
    \item \uline {{\bf Chiamata prenotazione}}
    \begin{itemize}
        \item {\bf Attore}: Operatore
        \item {\bf Input}: Tipo Prenotazione, Orario Prenotazione, Numero Posti
        \item {\bf Precondizioni}: Disponibilit\`a Tavoli
        \item {\bf Output}: Prenotazione presa in carico dal sistema
        \item {\bf PostCondizioni}: Conferma della prenotazione all'utente
        \item {\bf Descrizione scenari}:
        \begin{itemize}
            \item \uline{{\bf Scenario 1}}:
            \begin{itemize}
                \item L'utente chiama per la prenotazione di un tavolo
                \item L'operatore controlla la disponibilit\`a del tavolo
                \item L'operatore comunica la disponibilit\`a del tavolo all'utente
                \item L'utente conferma la prenotazione
            \end{itemize}
            \item \uline{{\bf Scenario 2}}:
            \begin{itemize}
                \item L'utente chiama per la prenotazione di un men\`u fisso
                \item L'operatore controlla la disponibilit\`a del men\`u
                \item L'operatore comunica la disponibilit\`a del men\`u all'utente
                \item L'utente conferma la prenotazione
            \end{itemize}
            \item \uline{{\bf Scenario 3}}
            \begin{itemize}
                \item L'utente chiama per la prenotazione di una sala
                \item L'operatore controlla la disponibilit\`a della sala
                \item L'operatore comunica la disponibilit\`a della sala
                \item L'utente concorda il men\`u con il proprietario
                \item L'utente conferma la prenotazione
            \end{itemize}
        \end{itemize}
    \end{itemize}
    \item \uline {{\bf Richiesta modifica ordine}}
    \begin{itemize}
        \item {\bf Attore}: Cameriere
        \item {\bf Input}: Numero Tavolo, Ordine da Modificare
        \item {\bf Precondizioni}: Nessuna
        \item {\bf Output}: Conferma di modifica
        \item {\bf Postcondizioni}: Nessuna
        \item {\bf Descizione Scenari}:
        \begin{itemize}
            \item \uline{Scenario 1}
            \begin{itemize}
                \item Il cliente richiede modifica di un ordine
                \item Il cameriere effettua la verifica 
            \end{itemize}
        \end{itemize}
    \end{itemize}
    \item \uline {{\bf Cancellazione ordine}}
    \item {\bf Caso d'uso}: \uline{{\bf Richiesta Conto}}
    \begin{itemize}
        \item {\bf Attore}: Operatore
        \item {\bf Input}: Numero Tavolo, Tipo di Conto
        \item {\bf Precondizioni}: Palmare operatore connesso alla rete
        \item {\bf Output}: Conferma di stampa conto
        \item {\bf Postcondizioni}: Nessuna
        \item {\bf Descrizione Scenari}:
        \begin{itemize}
            \item \uline{Scenario 1}:
            \begin{itemize}
                \item Il cliente richiede il conto
                \item L'operatore inserisce nel sistema il numero del tavolo e il 
                    tipo di conto richiesto
                \item Ottenute le informazioni, l'operatore le comunica al cliente
            \end{itemize}
            \item \uline{Scenario 2}:
            \begin{itemize}
                \item Il cliente richiede il conto
                \item L'operatore inserisce nel sistema il numero del tavolo e il 
                    tipo di conto richiesto
                \item Il conto non \`e disponibile a causa di un errore della rete
                \item L'operatore segnala un errore della sensori rete al sistema centrale
                \item L'operatore comunica l'impossibilit\`a di il conto in real time e richiede
                    le pietanze precedentemente ordinate.
            \end{itemize}
        \end{itemize}
    \end{itemize}
    \item \uline {{\bf Calcolo Retribuzioni Personale}}
    \begin{itemize}
        \item {\bf Attore}: Cassiere
        \item {\bf Input}: Nome Dipendente, Tipo di Qualifica, Tipo Contratto 
        \item {\bf Precondizioni}: Nessuna
        \item {\bf Output}: Conferma Operazione Eseguita
        \item {\bf Postcondizioni}: Nessuna
        \item {\bf Descrizione Scenari}:
        \begin{itemize}
            \item \uline{Scenario 1}:
            \begin{itemize}
                \item L'operatore inserisce nel sistema il nome del dipendente
                \item L'operatore richiede al sistema la paga della prestazione terminata
                \item Il sistema calcola l'ammontare della retribuzione
                \item L'operatore richiede la stampa del documento contabile
            \end{itemize}
        \end{itemize}
    \end{itemize}
    \item \uline {{\bf Gestione Personale}}
    \begin{itemize} 
        \item {\bf Attore}: Gestore Personale
        \item {\bf Input}: Nome Dipendente, Tipo di Qualifica, Tipo Contratto
        \item {\bf Precondizioni}: Nessuna
        \item {\bf Output}: Conferma Operazione Eseguita
        \item {\bf Postcondizioni}: Nessuna
        \item {\bf Descrizione Scenari}:
        \begin{itemize}
            \item \uline{Scenario 1}
            \begin{itemize}
                \item L'operatore inserisce nel sistema il nome del dipendente
                \item L'operatore richiede al sistema l'assunzione del dipendente
                \item Il sistema conferma l'operazione
            \end{itemize}
            \item \uline{Scenario 2}
            \begin{itemize}
                \item L'operatore inserisce nel sistema il nome del dipendente
                \item L'operatore richiede al sistema il licenziamento del dipendente
                \item Il sistema conferma l'operazione
            \end{itemize}
        \end{itemize}     
    \end{itemize}
    \item \uline {{\bf Redigi Bilancio}}
    \begin{itemize}
        \item {\bf Attore}: Cassiere
        \item {\bf Input}: Data Chiusura Bilancio, Periodo Copertura Bilancio
        \item {\bf Precondizioni}: Data Chiusura Bilancio $>$ di Data Apertura, Data di Chiusura $\leq$
            della data attuale
        \item {\bf Output}: Bilancio
        \item {\bf Postcondizioni}: Nessuna
        \item {\bf Descrizione Scenari}:
        \begin{itemize}
            \item \uline{Scenario 1}
            \begin{itemize}
                \item Il cassiere richiede il bilancio
                \item Il cassiere inserisce nel sistema la data di inzio e di termine del
                    periodo di valutazione del bilancio
                \item Il sistema calcola il bilancio e lo stampa a video
            \end{itemize}
            \item \uline{Scenario 2}
            \begin{itemize}
                \item Il cassiere richiede il bilancio
                \item Il cassiere inserisce nel sistema la data di inzio e di termine del
                    periodo di valutazione del bilancio
                \item Il sistema avverte l'utente con un messaggio di warning dell'errato
                    inserimento dei dati di Input
            \end{itemize}
            \item \uline{Scenario 3}
            \begin{itemize}
                \item Il cassiere richiede il bilancio
                \item Il cassiere inserisce nel sistema la data di inzio e di termine del
                    periodo di valutazione del bilancio
                \item Il sistema avverte l'utente che il database \`e corrotto
                \item Il sistema effettua un abort
            \end{itemize}
        \end{itemize}
    \end{itemize}
    \item \uline {{\bf Verifica Scorte}}
    \begin{itemize}
        \item {\bf Attore}: Operatore
        \item {\bf Input}: Magazzino da visionare
        \item {\bf Precondizioni}: Nessuna
        \item {\bf Output}: Itemlist delle scorte presenti
        \item {\bf Postcondizioni}: Le scorte devono sopperire ad almeno 2 giorni di mancata
            fornitura
        \item {\bf Descrizione degli Scenari}:
        \begin{itemize}
            \item \uline{Scenario 1}
            \begin{itemize}
                \item L'operatore richiede al sistema la visione delle scorte all'interno del magazzino
                \item Il sistema mostra la item list delle scorte
                \item L'operatore visiona l'item list
            \end{itemize}
            \item \uline{Scenario 2}
            \begin{itemize}
                \item Il sistema invia un messaggio di warning di "ammanchi" nel magazzino
                \item L'operatore richieste l'item list degli "ammanchi" al sistema 
                \item L'operatore effettua l'ordine per la fornitura
            \end{itemize}
        \end{itemize}
    \end{itemize} 
\end{enumerate}
